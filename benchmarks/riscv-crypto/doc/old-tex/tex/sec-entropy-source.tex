%\section{The Entropy Source Interface}
\label{sec:entropy-source}

    The proposed RISC-V TRNG ISA is primarily an Entropy Source (ES)
    interface. A valid implementation should satisfy properties that allow
    it to be used to \emph{seed} standard and nonstandard cryptographic
    DRBGs of virtually any state size and security level.

    The purpose of this baseline specification is to guarantee that a simple,
    device-independent driver component (e.g., in Linux kernel, embedded
    firmware, or a cryptographic library) can use the ISA instruction to
    generate truly random bits. See Appendix \ref{sec:entropy-appendix} for
    rationale and further discussion. This section is also supported by
    \cite{SaNeMa20}.

\subsection{PollEntropy Instruction}
\label{sec:es-pollentropy}

    The main ISA-level interface consists of a single pseudoinstruction,
    \mnemonic{pollentropy} that returns a 32/64-bit value in a CPU register.
    It is invoked in {\bf Machine Mode} (which may be the only mode) as
    follows:

\begin{cryptoisa}
RV32, RV64
    pollentropy    rd   // Poll randomness. Encoding: csrrs rd, mentropy, x0
\end{cryptoisa}
    The \mnemonic{pollentropy} pseudoinstruction reads XLEN bits from the
    {\tt mentropy} read-only machine-mode CSR
    described in Table \ref{tab:mentropy}.

    \begin{table}[h!]
    \begin{center}
    \begin{tabular}{ccl}
    \toprule
    Bits    & Name  & Description \\
    \midrule
    \verb|63:32|    & {\it Set to 0}
            & Upper bits are set to zero in RV64. \\
    \verb|31:30|    & \verb|OPST|
            & Status:   \verb|BIST| (00), \verb|ES16| (01),
                        \verb|WAIT| (10),   \verb|DEAD| (11). \\
    \verb|29:24|    & {\it reserved}
            & For future use by the RISC-V specification. \\
    \verb|23:16|    & {\it custom}
            & Reserved for custom and experimental use. \\
    \verb|15: 0|    & \verb|seed|
            & 16 bits of randomness, only when \verb|OPST=ES16|.    \\
    \bottomrule
    \end{tabular}
    \caption{
        The {\tt mentropy} CSR.
        It uses address {\tt 0xF15}, a standard read-only machine-mode CSR.
    }
    \label{tab:mentropy}
    \end{center}
    \end{table}

    The instruction is {\bf non-blocking} and returns immediately, either
    with two status bits \verb|mentropy[31:30]| = \verb|OPST| set to
    \verb|ES16| (01), indicating successful polling, or with {\bf no}
    entropy and one of three polling failure statuses \verb|BIST| (00),
    \verb|WAIT| (10), or \verb|DEAD| (11), discussed below.

    The sixteen bits of randomness in \verb|mentropy[15:0]|=\verb|seed| polled
    with \verb|ES16| status {\bf must be cryptographically conditioned}
    before they can be used as (up to 8 bits of) keying material. We suggest
    entropy output to be post-processed in blocks of at least 256 bits,
    with 128 bit resulting output block.

    When  \verb|OPST| is not \verb|ES16|, \verb|seed| should be set to 0.
    An implementation may safely set reserved and custom bits to zeros.
    A polling software interface should ignore their contents.

    The Status Bits at \verb|mentropy[31:30]|=\verb|OPST|:

\begin{itemize}
    \item[00]   \underline{\tt BIST}
    indicates that Built-In Self-Test ``on-demand'' (BIST) statistical
    testing is being performed. In typical implementations,
    \verb|BIST| will last only a few milliseconds, up to a few hundred.
    If the system returns temporarily to \verb|BIST| from any other state,
    this signals a non-fatal (usually non-actionable) self-test alarm.
    {\tt BIST} is also used to signal test mode (\mnemonic{getnoise}).

    \item[01]   \underline{\tt ES16}
    indicates success; the low bits \verb|mentropy[15:0]| will have 16 bits
    of randomness which must be guaranteed to have at least 8 bits true
    entropy regardless of implementation. For example, \verb|0x4000ABCD|
    is a valid \verb|ES16| status output on RV32, with \verb|0xABCD| being
    the \verb|seed| value.

    \item[10]   \underline{\tt WAIT}
    means that a sufficient amount of entropy is not yet available.
    This is not an error condition and may (in fact) be more frequent than
    ES16, since true entropy sources may not have very high bandwidth.
    If polling in a loop, we suggest calling \mnemonic{wfi} (wait for
    interrupt) before the next poll.

    \item[11]   \underline{\tt DEAD}
    is an unrecoverable self-test error. This may indicate a hardware
    fault, a security issue, or (extremely rarely) a type-1
    statistical false positive in the continuous testing procedures.
    Implementations do not need to implement \verb|DEAD| as it may not require
    an end-user notification; an immediate lock-down may be a more
    appropriate response in dedicated security devices.
\end{itemize}


\subsection{Polling Mechanism with WFI}
\label{sec:es-polling}

\begin{figure}[tb]
    \centering
    %   esstate.tikz
%   2020-06-26  Markku-Juhani O. Saarinen <mjos@pqshield.com>

\begin{tikzpicture}[>=latex,scale=1.2]
\tikzset{every state/.append style={rectangle, thick, rounded corners}};

%   NOISE_TEST = 0

\draw[draw=none,fill=gray!10] (-1.5,0) rectangle ++(3,0.5) node[pos=0.5]
    {\small\tt NOISE\_TEST = 0};

\node[align=center] (reset) at (0,-0.5) {reset};
\node[state,fill=cyan!10] (bist) at (0,-1.5) {BIST};

\node[state,fill=yellow!20] (wait) at (-1,-3) {WAIT};

\node[state,fill=green!20] (es16) at (1,-3) {ES16};

\node[align=center] (seed) at (2,-3) {{seed}\\{ok!}};

\node[state,fill=red!20] (dead) at (0,-4.5) {DEAD};

\node[align=center] (dead2) at (0,-5.5) {\em stays dead!};

\draw[->] (reset) to (bist);
\draw[->] (bist) to (wait);
\draw[->] (bist) to (es16);

\draw[->, bend left] (wait) to (es16);
\draw[->, bend left] (es16) to (wait);

\draw[color=red,dashed,->] (bist) .. controls (-2.5,-2.5) and (-2.5,-3.5) .. (dead);

\draw[color=red,dashed,->] (wait) to (dead);
\draw[color=red,dashed,->] (es16) to (dead);

\draw[->] (dead) .. controls ++(0.15,-0.85) and ++(-0.15,-0.85) .. (dead);

\node[color=red,align=center] at (-1.5,-4.5) {{\it fatal}\\{\it error}};

\draw[color=cyan,densely dotted,->] (bist) .. controls ++(-0.85,-0.15)
    and ++(-0.85,0.15) .. (bist);

\draw[color=cyan,densely dotted,->] (wait) .. controls (-1.0,-2.7)
    and (-1.0,-2.2) .. (bist);

\draw[color=cyan,densely dotted,->] (es16) .. controls (1.0,-2.7)
    and (1.0,-2.2) .. (bist);

\node[color=cyan,align=center] at (1.5,-2.0) {{\it non-fatal}\\{\it alarm}};

%   NOISE_TEST = 1

\draw[dashed,gray] (2.5,-1) -- (2.5,-5);

\draw[draw=none,fill=gray!10] (3.5,0) rectangle ++(3,0.5) node[pos=0.5]
    {\small\tt NOISE\_TEST = 1};

\node[state,fill=cyan!10] (bist2) at (5,-1.5) {BIST};

\draw[color=cyan,densely dotted,->] (bist2) .. controls ++(0.85,0.15)
    and ++(0.85,-0.15) .. (bist2);

\node[state,fill=red!20] (dead2) at (5,-4.5) {DEAD};

\draw[->] (dead2) .. controls ++(0.15,-0.85) and ++(-0.15,-0.85) .. (dead2);

\draw[color=red,dashed,->] (bist2) .. controls (7.5,-2.5) and (7.5,-3.5) .. (dead2);


\draw[gray,->] (bist2) .. controls (3,-0.7) and (2,-0.7) .. (bist)
    node[pos=.5,above] {\small test mode disabled};

\draw[gray,->] (dead2) .. controls (3,-5.2) and (2,-5.2) .. (dead)
    node[pos=.5,below] {\small test mode disabled};

\node[right,color=gray] (enable) at (2.5,-1.3) {\small\underline{enabled:}};
\node[right,color=gray] (bistn) at (2.5,-1.7) {\small BIST};
\node[right,color=gray] (waitn) at (2.5,-2.1) {\small WAIT};
\node[right,color=gray] (es16n) at (2.5,-2.5) {\small ES16};

\node[right,color=gray] (deadn) at (2.5,-4.5) {\small DEAD};

\draw[gray,->] (bistn.e) -- (bist2);
\draw[gray,->] (waitn.e) -- (bist2);
\draw[gray,->] (es16n.e) -- (bist2);
\draw[gray,->] (deadn.e) -- (dead2);

\node[state,dashed,align=center] (getnoise) at (5,-3)
    { {\small GetNoise active:}\\{\small {\bf no} PollEntropy}};

\end{tikzpicture}


    \caption{Normally the operational state alternates between WAIT
        (no data) and ES16, which means that 16 bits of randomness (seed)
        has been polled. BIST (Built-in Self-Test) only occurs after reset
        or to signal a non-fatal self-test alarm (if reached after WAIT or
        ES16). DEAD is an unrecoverable error state.
        In test mode (when GetNoise is active) WAIT and ES16 states are
        unavailable in PollEntropy.}
%   \Description{State diagram of the Entropy Source.}
    \label{fig:esstate_tikz}
\end{figure}

    Figure \ref{fig:esstate_tikz} illustrates operational state
    (\verb|OPST|) transitions. The state is usually either \verb|WAIT| or
    \verb|ES16|. There are no mandatory interrupts. However, the polling
    mechanism should be implemented in a way that allows even generic
    non-interrupt drivers to benefit from interrupts.

    We specifically recommend against busy-loop polling on this instruction
    as it may have relatively low bandwidth. Even though no specific interrupt
    sequence is specified, it is required that the \mnemonic{wfi} (wait for
    interrupt) instruction is available.
    Cores which implement \mnemonic{mentropy} must not raise an Illegal
    Instruction Exception when executing \mnemonic{wfi} unless required to
    do so by the Timeout Wait bit of the \mnemonic{mstatus} register,
    as detailed in Section 3.1.6.5 of the Privileged ISA Manual \cite{_WaAs19A}.
    The RISC-V ISA allows \mnemonic{wfi} to be implemented as a \mnemonic{nop}.
    As a minimum requirement for portable drivers, a \verb|WAIT| or
    \verb|BIST| from \mnemonic{pollentropy} should be followed by a
    \mnemonic{wfi} before another \mnemonic{pollentropy} instruction is
    issued. There is no need to poll after a \verb|DEAD| state.

    To guarantee that no sensitive data is read twice and that different
    callers don't get correlated output, it is suggested that hardware
    implements ``wipe-on-read'' on the randomness pathway during each read
    (successful poll). For the same reasons, only complete and fully
    processed randomness words shall be made available via
    \mnemonic{pollentropy} (no half-conditioned buffers or even full buffers
    in \verb|WAIT| state -- even if they are to be ignored by compliant
    callers).


\subsection{Entropy Source Requirements}
\label{sec:req-es}
\label{sec:req-entropy}
\label{sec:req-iid}
\label{sec:req-state}

    Output \verb|seed| from \mnemonic{pollentropy} is not necessarily fully
    conditioned randomness due to hardware limitations of smaller,
    low-powered implementations. However minimum requirements are
    defined. A caller should not use the output directly but poll
    twice the amount of required entropy, cryptographically condition
    (hash) it, and use that to seed a cryptographic DRBG.

    The expectation is that \verb|seed| output passes typical randomness
    tests, but pseudorandom conditioner may cause weak entropy sources to
    pass such tests as well. The results of such tests should not be
    confused with the security or robustness of an entropy source.
    Evaluation of entropy sources involves an investigation of the
    stochastic model of the noise source, an analysis of the conditioning
    component, continuous tests, etc. Entropy estimation (statistical
    testing) is just a part of this process.

    The specification of entropy source requirements is complicated by the
    existence of two slightly conflicting standards -- NIST SP 800-90B
    \cite{TuBaKe+18} for FIPS 140-3 and AIS 31 \cite{KiSc11} for Common
    Criteria evaluations. RISC-V hardware vendors may design their entropy
    sources to meet {\bf either one of these standards} (as different type
    of evidence is required for each certification). It is also possible for
    implementations to meet both criteria. Alternatively, for virtual
    entropy sources (DRBGs), the post-processed output must meet the
    ``256-bit security'' requirements of Category 5 post-quantum cryptography.


    \subsubsection{Virtual Entropy Sources -- Secret State Size Requirement}

    A \mnemonic{pollentropy} implementation can always output fully
    conditioned, perfectly distributed numbers. However, it is required
    that if a DRBG is used as a source, it must have an internal state with
    at least 256 bits of secret entropy (Example: a CTR\_DRBG built from
    AES-128 is not sufficient). Any implementation of \mnemonic{pollentropy}
    that limits the security strength shall not reduce it to less than
    256 bits.

    RISC-V requires drivers to implement at least 2-to-1 cryptographic
    post-processing in software with the expectation that the final output
    from this post-processing would should have ``computationally bounded
    full entropy''.  The computational bound is set so that a
    random-distinguishing attack should require computational resources
    comparable or greater than those required for key searching a block cipher
    with a 256-bit key (e.g., AES 256). This is equivalent to a Category 5
    classical or quantum adversary
    \cite[Section 4.A.4 Security Strength Categories]{NI16}.

    \subsubsection{FIPS 140-3 / SP 800-90 IID Track}

    For FIPS 140-3 certification, vendors should design their physical entropy
    sources so that they can be submitted to the  ``IID track''. This usually
    requires post-processing of noise in hardware.\footnote{NIST SP 800-90B
        \cite{TuBaKe+18} and its  FIPS 140-3 Implementation Guidance
        \cite[Section D.K.]{NICC20} are being revised at the time of writing.
        The type of appropriate post-processing for IID track depends on
        the physical noise source. Some simple post-processing methods are
        currently considered to be part of the sampling (``digitization'')\
        step of the process. Vetted or non-vetted conditioning may also be
        required in hardware, forming a hardware-software conditioning chain.}

    \begin{itemize}

    \item[\S E1]    {\bf Entropy Requirement.}
    Each 16-bit output sample (\verb|seed|) must have more than 8 bits of
    independent, unpredictable randomness. This minimum requirement is
    satisfied if (in a NIST SP 800-90B \cite{TuBaKe+18} assessment) 128
    bits of output entropy can be obtained from each 256-bit
    ($16 \times 16$-bit) \mnemonic{pollentropy} output sequence via a vetted
    cryptographic conditioning algorithm (see Section 3.1.5.1.2 in
    \cite{TuBaKe+18}). This means that the actual SP 800-90B entropy
    assessment must yield significantly more than 8 bits (vendors should
    aim at 16).

    Driver developers may make this conservative assumption but are not
    prohibited from using more than twice the number of seed bits relative
    to the desired resulting entropy.

    \item[\S E2]    {\bf SP 800-90B IID}
    The output must be close to \emph{Independent and Identically Distributed}
    (IID), meaning that the output distribution does not deteriorate over
    time and that output words convey little information about each other.
    This requirement is satisfied if the construction of the physical source
    and sampling mechanism suggests nothing against the IID assumption
    and the IID tests in Section 5 of NIST SP 800-90B \cite{TuBaKe+18} are
    consistently passed.

    \end{itemize}

    Note that both requirements must be satisfied (\S E1 may appear looser
    than \S E2). FIPS 140-3 certification of course imposes many additional
    requirements.


    \subsubsection{Common Criteria / AIS 31 PTG.2 Class}

    For alternative Common Criteria certification (or self-certification)
    vendors should target AIS 31 PTG.2 requirements \cite[Sect. 4.3.]{KiSc11}.
    Entropy sources (\verb|seed| bits) are viewed as ``internal random
    numbers'' in the context of AIS 31. Note that PTG.2 does not preclude
    other certification levels -- especially PTG.3 when combined with
    appropriate post-processing and DRBG on the software side.

    For validation purposes, the PTG.2 requirements may be mapped to security
    controls \S T 1-3 (Sect. \ref{sec:security-controls}) and the
    \mnemonic{pollentropy} interface as follows:

    \begin{itemize}
    \item[\S P1]{\bf [PTG.2.1]} Start-up tests map to \S T1 and reset-triggered
            (on-demand) \verb|BIST| tests.

    \item[\S P2]{\bf [PTG.2.2]} Continuous testing total failure maps to \S T2 and
        the \verb|DEAD| state.

    \item[\S P3]{\bf [PTG.2.3]}  Online tests are continuous tests of \S T2 --
        entropy output is prevented in the \verb|BIST| state.

    \item[\S P4][{\bf PTG.2.4]} Is related to the design of effective entropy source
        health tests, which we encourage.

    \item[\S P5]{\bf [PTG.2.5]} Raw random sequence may be checked via the
        GetNoise interface (Section \ref{sec:getnoise}).

    \item[\S P6][{\bf PTG.2.6]} Test Procedure A \cite[Sect 2.4.4.1]{KiSc11} is part of
        the evaluation process, and we suggest self-evaluation using these
        tests even if Common Criteria certification is not sought by a vendor.

    \item[\S P7]{\bf [PTG.2.7]} Average Shannon entropy of ``internal random
        bits'' exceeds 0.997.
    \end{itemize}

    Even though \S E1, \S E2, and post-processing imply that less than 16 of
    \verb|seed| bits may be designated as internal random bits for \S P7,
    we recommend that all 16 ES bits meet this requirement.

    Note that in Common Criteria validation the SP 800-90B IID requirement
    of \S E2 is not stated. However, it may also be satisfied -- and the
    H=0.997 level of \S P7 / PTG.2.7 leaves relatively little ``room'' for an
    entropy defect (mutual entropy).
    Also note that the SP 800-90B validation process is concerned with
    min-entropy, not Shannon entropy, so these numbers are not directly
    comparable.


\subsection{Security Controls (Tests)}
\label{sec:security-controls}

    The primary purpose of a cryptographic entropy source is to produce
    secret keying material. In almost all cases a hardware entropy source
    must implement appropriate \emph{security controls} to guarantee
    unpredictability, prevent leakage, detect attacks, and deny
    adversarial control over the entropy output or ts generation mechanism.
    Security controls are not mandatory for RISC-V (in case of virtual
    entropy sources) but are required for security certification.

    Many of the security controls built into the device are called ``health
    checks.'' Health checks can take the form of integrity checks, start-up
    tests, and on-demand tests. These tests can be implemented in hardware
    or firmware; typically both. Several are mandated by standards such as
    NIST SP 800-90B \cite{NI19}. The choice of appropriate health tests
    depends on the certification target, system architecture, the threat
    model, entropy source type, and other factors.

    Health checks are not intended for hardware diagnostics but for
    detecting security issues -- hence the default action should be aimed
    at damage control (prevent weak crypto keys from being
    generated). Additional ``debug'' mechanisms may be implemented if
    necessary, but then the device must be outside production use.

    \begin{itemize}

    \item[\S T1]    {\bf On-demand testing.}
    A sequence of simple tests is invoked via resetting, rebooting, or
    powering-up the hardware (not an ISA signal). The implementation will
    simply return  \verb|BIST| during the initial start-up self-test period;
    in any case, the driver must wait for them to finish before starting
    cryptographic operations. Upon failure the entropy source will enter a
    no-output \verb|DEAD| state.

    \item[\S T2]    {\bf Continuous checks.}
    If an error is detected in continuous tests or environmental sensors,
    the entropy source will enter a no-output state.  We define that a
    non-critical alarm is signaled if the entropy source returns
    to \verb|BIST| state from live (\verb|WAIT| or \verb|ES16|) states.
    Such a \verb|BIST| alarm should be latched until polled at least once.
    Critical failures will result in \verb|DEAD| state immediately.
    A hardware-based continuous testing mechanism must not make statistical
    information externally available, and it must be zeroized periodically or
    upon demand via reset, power-up, or similar signal.

    \item[\S T3]{\bf Fatal error states,}
    Since the security of most cryptographic operations depends on the
    entropy source, a system-wide ``default deny'' security policy approach
    is appropriate for most entropy source failures.
    A hardware test failure should at least result in the \verb|DEAD|
    state and possibly reset/halt. It's a show stopper: The entropy
    source (or its cryptographic client application) \emph{must not} be
    allowed to run if its secure operation can't be guaranteed.

    \end{itemize}

\subsection{GetNoise Test Interface}
\label{sec:getnoise}

    The optional GetNoise interface allows access to ``raw noise'' and
    is mainly intended for manufacturer tests and validation of security
    modules. It is must not be used as a source of randomness or for
    other production use. Its contents and behavior must be interpreted
    in the context of {\tt mvendorid}, {\tt marchid}, and {\tt mimpid}
    CSR identifiers, so \mnemonic{getnoise} is effectively ``custom''.

    The interface consists of the {\tt mnoise} machine-mode CSR, which
    (unlike {\tt mentropy}) is read-write. We define a
    pseudoinstruction for reading it:
\begin{cryptoisa}
RV32, RV64
    getnoise    rd   // Noise source test. Encoding: csrrs rd, mnoise, x0
\end{cryptoisa}

    The Crypto ISE defines the semantics of only a single bit,
    \verb|mnoise[31]|, which is named \verb|NOISE_TEST|. The only
    universal function is for enabling/disabling this interface. This is
    because the test interface effectively disables \mnemonic{pollentropy};
    this way a soft reset can also reset this feature.

    The {\tt mnoise} CSR uses address {\tt 0x7A9}, indicating it is a
    standard read-write machine-mode CSR.
    This places it adjacently to debug/trace CSRs, indicating that
    it is not expected to be used in production.

    When \verb|NOISE_TEST = 1| in \mnemonic{getnoise} and {\tt mnoise},
    \mnemonic{pollentropy} and {\tt mentropy} {\bf must not} return
    anything via \verb|ES16|; we recommend
    that it is in \verb|BIST| state. When \verb|NOISE_TEST| is
    again disabled, the entropy source shall return from \verb|BIST|
    via a zeroization and self-test mechanism (effectively a reset).

    When not implemented (e.g., in virtual machines), \mnemonic{getnoise}
    can permanently read zero (\verb|0x00000000|). When available, but
    \verb|NOISE_TEST = 0|, \mnemonic{getnoise} can return a nonzero
    constant such as \verb|0x00000001|.

    The behavior of other input and output bits is left to the vendor.
    Although not used in production, we recommend that the instruction
    is always non-blocking.

